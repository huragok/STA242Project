\section{Data Visualization on the Facebook Data}

\subsection{Data Set Description}
In this part of our homework, we focus on the visualization of the
Facebook data~\cite{facebook} for community detection in ego
networks~\cite{leskovec2012learning}.
The publication and the C++ code related to this work can be found at this
webpage~\cite{homepagemcauley}. The original data consists of 10 ego networks
originated from 10 different root users containing 50~800 users, each labeled into a few
(potentially overlapping) "circles". Also, the features of each user in the ego
network is available. The goal of~\cite{leskovec2012learning} is, given a single
ego network, to infer the circles and the members corresponding to each circle based on the
links and the features of each user in the ego network. 

\subsection{Framework}
To enable a dynamic, interactive visualization of the original dataset and the
community detection algorithm in~\cite{leskovec2012learning} using the source
code available at~\cite{homepagemcauley}, we build a light-weighted
Flask application. We process the original data in Python and generate
.json files and pass it to the webpages, which contains a piece of javascript
code using d3 library to generate .svg figures and some text information.

Our visualization mainly consists of 2 parts:
\begin{enumerate}
    \item Visualization of the raw data, as the circles in each ego network is
    well-labeled.
    \item Visualization of the community detection algorithm, where the user can
    set the expected number of communities.
\end{enumerate}
Each part of visualization focuses on 2 aspects:
\begin{enumerate}
    \item Find the hierarchical relationship in the labeled/deteted circles,
    i.e. which circles is fully contained in which circles. Since the
    hierarchical relationship can be represented with a directed acyclic graph
    (DAG), here we make use of the dagre-d3 library~\cite{dagred3} to plot the
    .svg figures.
    \item Plot the ego networks with force-directed graph (FDG), i.e.
    visualizing the information embedded in the ego network and the
    labels/community detection results. For this project we use d3's force
    directed layout functionalities. Though it is easy to implement, we noticed
    that it is extremely computationally intensive for relatively large ego
    networks. We refer to~\cite{kobourov2004force} and~\cite{ogdf} for a review
    and some implementations of more advanced FDG layout algorithms. The integration of
    more advanced FDG algorithm is left to our future works.
\end{enumerate}

\subsection{Examples}
Our visualization on the Facebook data starts with an index page where a brief
introduction (and a few disclaimers) are presented. Then the user gets to choose
which ego network to visualize, as shown in Fig.~\ref{fig:page1} (Here we pick
the ego network corresponding to user 414).

\begin{figure}[!t]
    \centering
    \includegraphics[width=150mm]{figs/page1.png}
    \caption{The index page.}
    \label{fig:page1}
\end{figure}

When submitted, a new tab is opened presenting the visualization of the original
data (Fig.~\ref{fig:page2}). This page mainly consists of 4 parts:
\begin{figure}[!t]
    \noindent\makebox[\textwidth][c]{
    \begin{minipage}[c]{1\linewidth}
      \centering
      \centerline{\includegraphics[width=1.0\linewidth]{figs/page2_12.png}}
      \centerline{(a) Summary and circle hierarchy.}\medskip
    \end{minipage}}
    \begin{minipage}[b]{1\linewidth}
      \centering
      \centerline{\includegraphics[width=1.0\linewidth]{figs/page2_3.png}}
      \centerline{(b) Force Directed Graph (FDG) of the ego network.}\medskip
    \end{minipage}
    \caption{Visualization of the original labeled data.}
    \label{fig:page2}
\end{figure}
\begin{enumerate}
    \item A brief summary on the orignal data: this ego network has 7 unique
    circles, 155 users and 3386 edges.
    \item The hierarchy graph of the circles. This figure is interactive in
    that when placing the mouse over each block, the number/indices of members
    in the corresponding circles will be shown. Also the figure can be zoomed
    and moved by scrolling/dragging.
    \item The FDG of the ego network. Users in different circles are colored
    differently (in accordance with the hierarchy graph), and the users in
    different numbers of circles (due to overlapping) are represented with
    different types of node as described in the legend. This figure is
    interactive in that 
    \begin{itemize}
        \item The nodes can be dragged to move under the influence of the
        forces.
        \item When placing the mouse over any node, the corresponding user ID
        and the circle IDs which contains this user will be shown.
        \item For user contained in multiple circles, its color can be
        toggled among the colors assigned to each of these circles by double
        clicking the corresponding node.
    \end{itemize}
    \item Finally, the user gets to run the community detection algorithm by
    specifying $K$, the number of expected circles (Here we pick $K=3$).
\end{enumerate}

The action in response to the client's submitting a $K$ is to looking for the
corresponding community detection results saved in a .out file. If it does not
exist, the server will run the community detection algorithm on the fly. Then
the detection results are processed into json files and a third page is rendered
using the same template as the second page (Fig.~\ref{fig:page3}).
Unfortunately, we can see that the detection results results are quite different
from the labeled data. In fact, our test show that the algorithm is highly
sensitive to its parameters.

\begin{figure}[!t]
    \centering
    \includegraphics[width=150mm]{figs/page3.png}
    \caption{Visualization of the community detection results.}
    \label{fig:page3}
\end{figure}